\documentclass[letter,twoside,12pt]{article}
\usepackage{lmodern}
\usepackage[T1]{fontenc}
\usepackage[spanish]{babel}
\usepackage[utf8]{inputenc}
\usepackage{amsmath}
\usepackage{amssymb}
\usepackage{amsthm}
\usepackage{fullpage}
\usepackage{latexsym}
\usepackage{enumerate}
\usepackage{enumitem}
\PassOptionsToPackage{hyphens}{url}\usepackage{hyperref}
\title{Algebra Lineal II: Taller}
\newtheorem{lemma}{Lema}
\author{Jonathan Andrés Niño Cortés}
\begin{document}
\maketitle

\begin{itemize}
\item Dividiendo $ P_f(x) $ por $ P_{f,min}(x) $ obtenemos dos polinomios $ Q(x) $ y $ R(x) $ tales que 

\begin{equation}
P_f(x)=Q(x)P_{f,min}(x)+R(x) \nonumber
\end{equation} 

y $ deg(R)<deg(P_{f,min}) $.

Ahora si evaluamos estos polinomios tenemos que $ P_f(f)=0 $ por el teorema de Calley-Hamilton y $ P_{f,min}=0 $ por la definición del polinomio minimal. Concluimos por lo tanto que $ R(f)=0 $ pero para que no contradiga la minimalidad del polinomio local concluimos que $ R(x)=0 $. Por lo tanto, el polinomio caracteristico es un multiplo del polinomio minimal.

\item La matriz asociada a esta transformación es

\begin{equation}
A=\begin{pmatrix}
\lambda & 1
\\0 & \lambda
\end{pmatrix} \nonumber
\end{equation}

Y el polinomio característico se calcula como

\begin{equation}
\begin{vmatrix}
t-\lambda & -1
\\0 & t- \lambda
\end{vmatrix} = (t-\lambda)^2 \nonumber
\end{equation}.

Por el punto anterior, el polinomio minimal en este caso puede ser $ t-\lambda $ o $ (t-\lambda)^2 $. Pero para descartar el primer caso podemos tomar por ejemplo el vector $ (0,1) $. Tenemos que $ (f-\lambda Id)(0,1)=(1,\lambda)-(0,\lambda)=(1,0) \not = (0,0)$. Por lo tanto, el polinomio minimal es $ (t-\lambda)^2 $.

\item Esta transformación es precisamente multiplicación por $ \lambda $, es decir, $ f= \lambda Id $. El polinomio caracteristico de esta transformación es el mismo que el anterior $ (t- \lambda)^2 $ pero en este caso el polinomio minimal si es $ t-\lambda $. Esto porque $ f-\lambda Id = \lambda Id - \lambda_Id = 0 $. Esta transformación es semi-simple porque cualquier subespacio $ V $ es invariante, pues $ f(V)= \lambda V \subseteq V $.  

\end{itemize}
\end{document}