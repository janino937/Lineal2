\documentclass[11pt,a4paper]{article}
\usepackage{lmodern}
\usepackage[T1]{fontenc}
\usepackage[spanish]{babel}
\usepackage[utf8]{inputenc}
\usepackage{amsmath}
\usepackage{amsfonts}
\usepackage{amssymb}
\usepackage{amsthm}
\usepackage{graphicx}
\author{Jonathan Niño}
\title{Taller de Álgebra Lineal}

\newtheorem{thm}{Teorema}
\newtheorem{lem}[thm]{Lema}

\theoremstyle{definition}
\newtheorem{defn}{Definición}[section]
\newtheorem{conj}{Conjetura}[section]
\newtheorem{exmp}{Ejemplo}[section]
\theoremstyle{remark}
\newtheorem*{rem}{Observación}
\newtheorem*{note}{Nota}
\newtheorem{case}{Caso}
\newtheorem{exc}{Ejercicio}
\newtheorem{prop}{Proposición}

\newcommand{\abs}[1]{ \left\lbrace #1 \right\rbrace }
\newcommand{\set}[1]{\left\lbrace #1 \right\rbrace}
\newcommand{\RR}{\mathbb{R}}
\newcommand{\Hom}{\operatorname{Hom}}
\newcommand{\CC}{\mathbb{C}}
\begin{document}
	\maketitle
	Para esta tarea, utilizamos las siguientes definiciones:
	
	\begin{defn}
		Decimos que $ f $ es \textit{semi-simple} si para todo $ V_1 \leq V $ invariante bajo $ f $ existe $ V_2 \leq V$ invariante bajo $ f $ tal que $ V = V_1 \oplus V_2  $.
	\end{defn}
	
	\begin{defn}
		El \emph{polinomio minimal } de $ f $, $ P_{f, \min}(t) \in K[t] $ es el polinomio de menor grado tal que $ P_{f,\min}(f) = 0 $
	\end{defn}
	
\begin{exc}
	Demuestre que el polinomio minimal divide al polinomio característico. (\textit{Ayuda}: usando el algoritmo de la división, divida el polinomio característico por el polinomio minimal para obtener un residuo y verifique que este es igual a cero. )
\end{exc}
	
\begin{proof}
	Suponga que
\end{proof}	
	
\begin{exc}
	Sea $ \lambda \in K $ y defina $ f \in \Hom_K(K^2,K^2) $ por:
	\[ f(x,y) = (\lambda x + y, \lambda y) \]
	Verifique que $ f $ no es semi-simple. Encuentre el polinomio minimal de $ f $
\end{exc}	

\textit{Solución}

\begin{exc}
	Sea $ \lambda \in K $ y defina $ f \in \Hom_K(K^2, K^2) $ por:
	\[ f(x,y) = (\lambda x, \lambda y) \]
	Encuentre el polinomio minimal de $ f $. Verifique que $ f $ es semi-simple.
\end{exc}

\textit{Solución}

\begin{exc}
	Considere los operadores $ g,h \in \Hom_\RR (\RR^4, \RR^4) $ definidos por:
	\begin{align*}
	g(x,y,z,w) = \left( \frac{\sqrt{3}}{2}x + \frac{1}{2}w, \frac{\sqrt{3}}{2}y + \frac{1}{2}z, -\frac{1}{2}y + \frac{\sqrt{3}}{2}z, - \frac{1}{2}x + \frac{\sqrt{3}}{2} w \right) \\
	h(x,y,z,w) = \left( \frac{\sqrt{3}}{2}x + \frac{1}{2}z, \frac{\sqrt{3}}{2}y +z - \frac{1}{2}w, \frac{1}{2}x-y+ \frac{\sqrt{3}}{2}z, - \frac{1}{2}y + \frac{\sqrt{3}}{2}w \right)	\end{align*}
	y los operadores $ g_\CC $ y $  H_\CC \in \Hom_\CC ( \CC^4, \CC^4) $ dadas por las mismas fórmulas.
	\begin{enumerate}
		\item Verifique que $ g $ y $ h $ no tiene valores propios y que los valores propios de $ g_\CC $ y $ h_\CC $ son $ \lambda = \frac{1}{2}(\sqrt{3} + i-) $ y $ \bar{\lambda} = \frac{1}{2}(\sqrt{3} - i) $.
		\item Verifique que si $ u + iv $, con $ u,v \in \RR^4 $, es un vector propio de $ g_\CC $ asociado a $ \lambda $, entonces $ u - iv $ es un vector propio de $ g_\CC $ asociado a $ \bar{\lambda} $
		\item Encuentre una base de Jordan para $ g_\CC $ y para $ h_\CC $ de la forma $ T = \set{u_1 + i v_1, u_2 + i v_2, u_1 - i v_1, u_2 - iv_2 } $
		\item Usando la notación de los vectores en el numeral anterior, encuentre la representación matricial de $ g $ y $ h $ relativas a la base $ S = \set{u_1, - v_1, u_2, -v_2} $ 
		\item Encuentre los polinomios minimales de $ g $ y $ h $
	\end{enumerate}
	\textit{Solución}
	\begin{enumerate}
	\item La matriz asociada al operador $ g $ es 
	\begin{equation}
	A= \begin{pmatrix}
	\frac{\sqrt{3}}{2} & 0 & 0 & \frac{1}{2}
	\\ 0 & \frac{\sqrt{3}}{2} & \frac{1}{2} & 0 
	\\ 0 & -\frac{1}{2} & \frac{\sqrt{3}}{2} & 0
	\\ -\frac{1}{2} & 0 & 0 & \frac{\sqrt{3}}{2}
	\end{pmatrix} \nonumber
	\end{equation}
	
	Y el polinomio característico de $ g $ es
	\begin{multline}
	\begin{vmatrix}
	t-\frac{\sqrt{3}}{2} & 0 & 0 & -\frac{1}{2}
		\\ 0 & t-\frac{\sqrt{3}}{2} & -\frac{1}{2} & 0 
		\\ 0 & \frac{1}{2} & t-\frac{\sqrt{3}}{2} & 0
		\\ \frac{1}{2} & 0 & 0 & t-\frac{\sqrt{3}}{2}
	\end{vmatrix}\nonumber = (t- \frac{\sqrt{3}}{2})
	\begin{vmatrix}
			 t-\frac{\sqrt{3}}{2} & -\frac{1}{2} & 0 
			\\\frac{1}{2} & t-\frac{\sqrt{3}}{2} & 0
			\\0 & 0 & t-\frac{\sqrt{3}}{2}
		\end{vmatrix} \\
		-(-\frac{1}{2})\begin{vmatrix}
			0 & t-\frac{\sqrt{3}}{2} & -\frac{1}{2}
				\\ 0 & \frac{1}{2} & t-\frac{\sqrt{3}}{2}
				\\ \frac{1}{2} & 0 & 0
			\end{vmatrix}\nonumber = (t-\frac{\sqrt{3}}{2})^2\begin{vmatrix}
						 t-\frac{\sqrt{3}}{2} & -\frac{1}{2}
						\\\frac{1}{2} & t-\frac{\sqrt{3}}{2}
					\end{vmatrix}+(\frac{1}{2})^2\begin{vmatrix}
								t-\frac{\sqrt{3}}{2} & -\frac{1}{2}
									\\ \frac{1}{2} & t-\frac{\sqrt{3}}{2}
								\end{vmatrix}
	\\=(t-\frac{\sqrt{3}}{2})^2((t-\frac{\sqrt{3}}{2})^2+(\frac{1}{2})^2)+(\frac{1}{2})^2((t-\frac{\sqrt{3}}{2})^2+(\frac{1}{2})^2)
	\\=((t-\frac{\sqrt{3}}{2})^2+(\frac{1}{2})^2)((t-\frac{\sqrt{3}}{2})^2+(\frac{1}{2})^2)=((t-\frac{\sqrt{3}}{2})^2+(\frac{1}{2})^2)^2
	\end{multline}
	
	Vemos que este polinomio no tiene raíces en $ \mathbb{R} $ porque $ (t-\frac{\sqrt{3}}{2})^2 $ es mayor o igual a 0. Luego, $ (t-\frac{\sqrt{3}}{2})^2 + (\frac{1}{2})^2$ es estrictamente mayor que 0 y también su cuadrado.
	
    En $ \mathbb{C} $ este polinomio si tiene raíces. La descomposición esta dada por la fórmula para factorizar resta de cuadrados.
    \begin{equation}
    ((t-\frac{\sqrt{3}}{2})^2+(\frac{1}{2})^2)^2=((t-\frac{\sqrt{3}}{2})^2-(i\frac{1}{2})^2)^2=(t-\frac{\sqrt{3}}{2}-i\frac{1}{2})^2(t-\frac{\sqrt{3}}{2}+i\frac{1}{2})^2\nonumber
    \end{equation}
    Concluimos que $ \frac{\sqrt{3}}{2}+i\frac{1}{2} $ y $ \frac{\sqrt{3}}{2}-i\frac{1}{2} $ son las dos raices del polinomio caracteristico.
    
    Para el operador $ h $ la matriz asociada es \begin{equation}
    	B= \begin{pmatrix}
    	\frac{\sqrt{3}}{2} & 0 & \frac{1}{2} & 0
    	\\ 0 & \frac{\sqrt{3}}{2} & 1 & -\frac{1}{2} 
    	\\ \frac{1}{2} & -1 & \frac{\sqrt{3}}{2} & 0
    	\\ 0 & -\frac{1}{2} & 0 & \frac{\sqrt{3}}{2}
    	\end{pmatrix} \nonumber
    	\end{equation}
	
	El polinomio característico es
		
		\begin{multline}
		    	\begin{vmatrix}
		    	t-\frac{\sqrt{3}}{2} & 0 & -\frac{1}{2} & 0
		    	\\ 0 & t-\frac{\sqrt{3}}{2} & -1 & \frac{1}{2} 
		    	\\ -\frac{1}{2} & 1 & t-\frac{\sqrt{3}}{2} & 0
		    	\\ 0 & \frac{1}{2} & 0 & t-\frac{\sqrt{3}}{2}
		    	\end{vmatrix} = (t-\frac{\sqrt{3}}{2})
		    	\begin{vmatrix}
	 		    	t-\frac{\sqrt{3}}{2} & -1 & \frac{1}{2} 
	 		    	\\ 1 & t-\frac{\sqrt{3}}{2} & 0
	 		    	\\ \frac{1}{2} & 0 & t-\frac{\sqrt{3}}{2}
	 		    \end{vmatrix}\nonumber
	 		    \\+(-\frac{1}{2})\begin{vmatrix}
	 		    	    	0 & t-\frac{\sqrt{3}}{2} & \frac{1}{2} 
	 		    	    	\\ -\frac{1}{2} & 1 & 0
	 		    	    	\\ 0 & \frac{1}{2} &  t-\frac{\sqrt{3}}{2}
	 		    	    	\end{vmatrix} = (t-\frac{\sqrt{3}}{2})(\frac{1}{2}\begin{vmatrix}
	 		    	    	 		    	 -1 & \frac{1}{2} 
	 		    	    	 		    	\\t-\frac{\sqrt{3}}{2} & 0
	 		    	    	 		    \end{vmatrix}+(t-\frac{\sqrt{3}}{2})\begin{vmatrix}
	 		    	    	 		     		    	t-\frac{\sqrt{3}}{2} & -1 
	 		    	    	 		     		    	\\ 1 & t-\frac{\sqrt{3}}{2}
	 		    	    	 		     		    \end{vmatrix})\\
	 		 +(-\frac{1}{2})(\frac{1}{2})\begin{vmatrix}
	 		  		    	    	t-\frac{\sqrt{3}}{2} & \frac{1}{2}
	 		  		    	    	\\ \frac{1}{2} &  t-\frac{\sqrt{3}}{2}
	 		  		    	    	\end{vmatrix} = (t-\frac{\sqrt{3}}{2})(\frac{1}{2}(t-\frac{\sqrt{3}}{2})(-\frac{1}{2})+(t-\frac{\sqrt{3}}{2})((t-\frac{\sqrt{3}}{2})^2+1))\\
	 		+(-\frac{1}{2})(\frac{1}{2})((t-\frac{\sqrt{3}}{2})^2-(\frac{1}{2})^2)= (t-\frac{\sqrt{3}}{2})^2((t-\frac{\sqrt{3}}{2})^2+\frac{3}{4})-(\frac{1}{2})^2 ((t-\frac{\sqrt{3}}{2})^2-\frac{1}{4})\\
	 		=(t-\frac{\sqrt{3}}{2})^2((t-\frac{\sqrt{3}}{2})^2+\frac{1}{4}+\frac{1}{2})-(\frac{1}{2})^2 ((t-\frac{\sqrt{3}}{2})^2+\frac{1}{4}-\frac{1}{2})
	 		\\ = (t-\frac{\sqrt{3}}{2})^2((t-\frac{\sqrt{3}}{2})^2+\frac{1}{4})-(\frac{1}{2})^2 ((t-\frac{\sqrt{3}}{2})^2+\frac{1}{4})+\frac{1}{2}(t-\frac{\sqrt{3}}{2})^2+(\frac{1}{2})^2\frac{1}{2}
	 		\\ = ((t-\frac{\sqrt{3}}{2})^2+\frac{1}{4})((t-\frac{\sqrt{3}}{2})^2-(\frac{1}{2})^2+\frac{1}{2})=((t-\frac{\sqrt{3}}{2})^2+\frac{1}{4})((t-\frac{\sqrt{3}}{2})^2+\frac{1}{4}) \\
	 		=((t-\frac{\sqrt{3}}{2})^2+\frac{1}{2}^2)^2\end{multline}
	 		
	 		Así que es el mismo polinomio característico que para el operador $ g $ y por lo tanto al igual que este, no tiene valores propios en $ \mathbb{R} $ y en $ \mathbb{C} $ son los mismos valores propios.
	 		
	 		\item 
	 		
	 		Primero demostremos que para cualquier $ z \in \mathbb{C} $, $ g_\mathbb{C}(\overline{z})=g_\mathbb{C}\overline{f(z)} $, esto es así porque todos los términos de la matriz de $ g_\mathbb{C} $ son reales. Notese que $ h_\mathbb{C} $ también cumple esta condición por lo que lo anterior también vale para $ h_\mathbb{C} $.
	 		
	 		Sea $ z = u+iv $ con $ u,v \in \mathbb{R} $. Por linealidad tenemos que $ g_\mathbb{C}(z)=g_\mathbb{C}(u)+ig_\mathbb{C}(v) $, pero como $ u $ y $ v $ son reales $ g_\mathbb{C}(u) = g(u) \in \mathbb{R} $ y $ g_\mathbb{C}(v) = g(v) \in \mathbb{R} $. Por lo tanto, $ g(u) $ es la parte real de $ g_\mathbb{C}(z) $ y $ g(v) $ es la parte imaginaria de $ g_\mathbb{C}(z) $.
	 		
	 		Ahora, el conjugado de $ z $ es $ u-iv $. $ g_\mathbb{C}(\overline{z})=g_\mathbb{C}(u)-ig_\mathbb{C}(v)=g(u)-ig(v) $. Vemos que la parte real es la misma que la anterior y la parte imaginaria es el opuesto aditivo de la anterior. Por lo tanto, es el conjugado de la anterior, es decir, $ \overline{g(z)} $.
	 		
	 		Ahora sea $ z=u+iv $ un vector propio asociado a $ \lambda=\alpha+i\beta $. Entonces, $ \overline{z}=u-iv $ es un vector propio asociado a $  \overline{\lambda} $.
	 		La primera oración se traduce en que $ f(z)=\lambda z $. Por propiedades de conjugación tenemos que $ \overline{\lambda z}=\overline{z}\overline{\lambda} $. Luego $ f(\overline{z})=\overline{f(z)}= \overline{\lambda z}=\overline{z}\overline{\lambda} $. Por lo que vemos que $ \overline{z} $ es el vector propio asociado a $ \overline{\lambda} $. 
			
			\item Para hallar una base canónica de Jordan para $ g_\mathbb{C} $ primero necesitamos calcular los vectores propios asociados a cada valor propio. Para esto primero necesitamos calcular el kernel de $ \lambda Id-g_\mathbb{C}$. La matriz asociada es
			
			\begin{equation}
			\begin{pmatrix}
			\lambda-\frac{\sqrt{3}}{2} & 0 & 0 & -\frac{1}{2}				\\ 0 & \lambda-\frac{\sqrt{3}}{2} & -\frac{1}{2} & 0 
								\\ 0 & \frac{1}{2} & \lambda-\frac{\sqrt{3}}{2} & 0
								\\ \frac{1}{2} & 0 & 0 & \lambda-\frac{\sqrt{3}}{2}
						\end{pmatrix} =
						\begin{pmatrix}
						i\frac{1}{2} & 0 & 0 & -\frac{1}{2}
								\\ 0 & i\frac{1}{2} & -\frac{1}{2} & 0 
								\\ 0 & \frac{1}{2} & i\frac{1}{2} & 0
								\\ \frac{1}{2} & 0 & 0 & i\frac{1}{2}
						\end{pmatrix}\nonumber
						\end{equation}
						
						Reduciendo por Gauss-Jordan obtenemos
						\begin{multline}
						\begin{pmatrix}
						i\frac{1}{2} & 0 & 0 & -\frac{1}{2}
						\\ 0 & i\frac{1}{2} & -\frac{1}{2} & 0 
						\\ 0 & \frac{1}{2} & i\frac{1}{2} & 0
						\\ \frac{1}{2} & 0 & 0 & i\frac{1}{2}
						\end{pmatrix} \Rightarrow
						\begin{pmatrix}
						i & 0 & 0 & -1
						\\ 0 & i & -1 & 0 
						\\ 0 & 1 & i & 0
						\\ 1 & 0 & 0 & i
						\end{pmatrix} \Rightarrow
						\begin{pmatrix}
						1 & 0 & 0 & i
						\\ 0 & 1 & i & 0
						\\ 0 & i & -1 & 0 
						\\ i & 0 & 0 & -1
						\end{pmatrix}
						\Rightarrow
						\\ \begin{pmatrix}
						1 & 0 & 0 & i
						\\ 0 & 1 & i & 0
						\\ 0 & -1 & -i & 0 
						\\ -1 & 0 & 0 & -i
						\end{pmatrix}
						\Rightarrow
						\begin{pmatrix}
						1 & 0 & 0 & i
						\\ 0 & 1 & i & 0
						\\ 0 & 0 & 0 & 0 
						\\ 0 & 0 & 0 & 0
						\end{pmatrix}
						 \nonumber		
						\end{multline}
						
						Entonces vemos que el kernel es de dimensión 2, es decir que tiene dos vectores asociados. Estos vectores son $ (1, 0,0,i) $ y $ (0,1,i,0) $. Por el punto anterior los vectores propios asociados a $ \overline{\lambda} $ serían $ (1, 0,0,-i) $ y $ (0,1,-i,0) $. Por lo tanto si tomamos $ u_1 = (1,0,0,0) $, $ u_2=(0,1,0,0) $, $ v_1=(0,0,0,1) $ y $ v_2 = (0,0,1,0) $ la base canónica de Jordan sería precisamente $ T = \set{u_1 + i v_1, u_2 + i v_2, u_1 - i v_1, u_2 - iv_2 } $ .
						
						Para $ h_\mathbb{C} $ debemos calcular el kernel de $ \lambda Id-h_\mathbb{C} $. La matriz asociada es
						
						\begin{equation}
						\begin{pmatrix}
				    	\lambda-\frac{\sqrt{3}}{2} & 0 & -\frac{1}{2} & 0
				    	\\ 0 & \lambda-\frac{\sqrt{3}}{2} & -1 & \frac{1}{2} 
				    	\\ -\frac{1}{2} & 1 & \lambda-\frac{\sqrt{3}}{2} & 0
				    	\\ 0 & \frac{1}{2} & 0 & \lambda-\frac{\sqrt{3}}{2}
				    	\end{pmatrix} = 
				    	\begin{pmatrix}
				    	i\frac{1}{2} & 0 & -\frac{1}{2} & 0
				    	\\ 0 & i\frac{1}{2} & -1 & \frac{1}{2} 
				    	\\ -\frac{1}{2} & 1 & i\frac{1}{2} & 0
				    	\\ 0 & \frac{1}{2} & 0 & i\frac{1}{2}
				    	\end{pmatrix} \nonumber
						\end{equation}
					
					De nuevo reducimos mediante Gauss-Jordan para obtener 
					\begin{multline}
					\begin{pmatrix}
			    	i\frac{1}{2} & 0 & -\frac{1}{2} & 0
			    	\\ 0 & i\frac{1}{2} & -1 & \frac{1}{2} 
			    	\\ -\frac{1}{2} & 1 & i\frac{1}{2} & 0
			    	\\ 0 & \frac{1}{2} & 0 & i\frac{1}{2}
			    	\end{pmatrix}  \Rightarrow
			    	\begin{pmatrix}
			    	1 & 0 & i & 0
			    	\\ 0 & 1 & 2i & -i 
			    	\\ i & -2i & 1 & 0
			    	\\ 0 & -i & 0 & 1
			    	\end{pmatrix}  \Rightarrow
			    	\begin{pmatrix}
			    	1 & 0 & i & 0
			    	\\ 0 & 1 & 2i & -i 
			    	\\ -1 & 2 & i & 0
			    	\\ 0 & -1 & 0 & -i
			    	\end{pmatrix}  \Rightarrow \\
			    	\begin{pmatrix}
			    	1 & 0 & i & 0
			    	\\ 0 & 1 & 2i & -i 
			    	\\ 0 & 2 & 2i & 0
			    	\\ 0 & 0 & 2i & -2i
			    	\end{pmatrix}  \Rightarrow
			    	\nonumber
			    	\begin{pmatrix}
			    	1 & 0 & i & 0
			    	\\ 0 & 1 & 2i & -i 
			    	\\ 0 & 0 & -2i & 2i
			    	\\ 0 & 0 & 2i & -2i
			    	\end{pmatrix}  \Rightarrow
			    	\begin{pmatrix}
			    	1 & 0 & i & 0
			    	\\ 0 & 1 & 2i & -i 
			    	\\ 0 & 0 & 1 & -1
			    	\\ 0 & 0 & 0 & 0
			    	\end{pmatrix}\Rightarrow
			    	\\\begin{pmatrix}
			    	1 & 0 & 0 & i
			    	\\ 0 & 1 & 0 & i 
			    	\\ 0 & 0 & 1 & -1
			    	\\ 0 & 0 & 0 & 0
			    	\end{pmatrix}
					\end{multline}
					
					En este caso el kernel es solo de una dimensión y es el generado por el vector $ (-i,-i,1,1) $.
					
					Ahora necesitamos calcular el kernel de $ (\lambda Id-h_\mathbb{C})^2 $ . La matriz asociada es
					
	\begin{equation}
		\begin{pmatrix}
	  	i\frac{1}{2} & 0 & -\frac{1}{2} & 0
	  	\\ 0 & i\frac{1}{2} & -1 & \frac{1}{2} 
	  	\\ -\frac{1}{2} & 1 & i\frac{1}{2} & 0
    	\\ 0 & \frac{1}{2} & 0 & i\frac{1}{2}
    	\end{pmatrix}
    	\begin{pmatrix}
    	i\frac{1}{2} & 0 & -\frac{1}{2} & 0
    	\\ 0 & i\frac{1}{2} & -1 & \frac{1}{2} 
    	\\ -\frac{1}{2} & 1 & i\frac{1}{2} & 0
    	\\ 0 & \frac{1}{2} & 0 & i\frac{1}{2}
    	\end{pmatrix} =
    	\begin{pmatrix}
    	0 & -\frac{1}{2} & -i\frac{1}{2} & 0
    	\\ \frac{1}{2} & -1 & -i & i\frac{1}{2} 
    	\\ -i\frac{1}{2} & i & -1 & \frac{1}{2}
    	\\ 0 & i\frac{1}{2} & -\frac{1}{2} & 0
    	\end{pmatrix} \nonumber
		\end{equation}
		
		Y para averiguar su kernel reducimos esta matriz usando Gauss-Jordan.
		
		\begin{multline}
		\begin{pmatrix}
    	0 & -\frac{1}{2} & -i\frac{1}{2} & 0
    	\\ \frac{1}{2} & -1 & -i & i\frac{1}{2} 
    	\\ -i\frac{1}{2} & i & -1 & \frac{1}{2}
    	\\ 0 & i\frac{1}{2} & -\frac{1}{2} & 0
    	\end{pmatrix} \Rightarrow
		\begin{pmatrix}
    	 1 & -2 & -2i & i 
    	\\0 & 1 & i & 0
    	\\ -1 & 2 & 2i & -i
    	\\ 0 & -1 & -i & 0
    	\end{pmatrix} \nonumber \Rightarrow
    	\begin{pmatrix}
   	 	1 & 0 & 0 & i 
    	\\0 & 1 & i & 0
    	\\ 0 & 0 & 0 & 0
    	\\ 0 & 0 & 0 & 0
 	    	\end{pmatrix}
		\end{multline}
		
		Entonces vemos que el vector $ (-i,0,0,1) $ es un vector del segundo nivel pues no esta incluido en el kernel anterior. Y $ (\lambda Id - h_\mathcal{C})(-i,0,0,1) = (\frac{1}{2},\frac{1}{2},i\frac{1}{2},i\frac{1}{2}) $ sería el segundo vector de la base de Jordan. 
		
		Por el segundo punto tenemos que $ (\frac{1}{2},\frac{1}{2},-i\frac{1}{2},-i\frac{1}{2}) $ esta en el kernel de $ (\overline{\lambda} Id - h_\mathcal{C}) $. Ahora para calcular un vector del segundo nivel tenemos que calcular el kernel de $ (\overline{\lambda} Id - h_\mathcal{C})^2 $.
		
		Pero observemos que $ (i,0,0,1) $ está en este kernel y no en el kernel anterior.
		
		\begin{equation}
		(\overline{\lambda} Id - h_\mathcal{C})(i,0,0,1) = 
		\begin{pmatrix}
	  	-i\frac{1}{2} & 0 & -\frac{1}{2} & 0
	  	\\ 0 & -i\frac{1}{2} & -1 & \frac{1}{2} 
	  	\\ -\frac{1}{2} & 1 & -i\frac{1}{2} & 0
    	\\ 0 & \frac{1}{2} & 0 & -i\frac{1}{2}
    	\end{pmatrix}
    	\begin{pmatrix}
	  	i
	  	\\ 0  
	  	\\  0
    	\\ 1
    	\end{pmatrix} = 
    	\begin{pmatrix}
	  	\frac{1}{2}
	  	\\ \frac{1}{2}  
	  	\\  -i\frac{1}{2}
    	\\ -i\frac{1}{2}
    	\end{pmatrix}
    	 \nonumber
		\end{equation}
		
		Y tenemos que $ (\overline{\lambda} Id - h_\mathcal{C})(\frac{1}{2},\frac{1}{2},-i\frac{1}{2},-i\frac{1}{2})=(0,0,0,0) $ por el punto anterior. Luego si tomamos $ u_1 = (0,0,0,1) $, $ v_1 = (1,0,0,0) $, $ u_2 = (\frac{1}{2}, \frac{1}{2},0,0)$ y $ v_2=(0,0,\frac{1}{2},\frac{1}{2}) $. La base $ T = \set{u_1 + i v_1, u_2 + i v_2, u_1 - i v_1, u_2 - iv_2 } $ es una base canónica de Jordan.
		
		\item Primero calculemos la matriz de transformación de la nueva base sugerida a la base canónica. Para $ g $ esta sería.
		
		\begin{equation}
		[id]_{S}^{can}= 
		\begin{pmatrix}
		1 & 0 & 0 & 0
		\\ 0 & 0 & 1 & 0  
		\\ 0 & 0 & 0 & -1
		\\ 0 & -1 & 0 & 0
		\end{pmatrix} \nonumber
		\end{equation}
		
		La matriz inversa $ [id]_{can}^{S} $ la podemos calcular usando Gauss-Jordan.
		
		\begin{multline}
		\begin{bmatrix}
		1 & 0 & 0 & 0
		\\ 0 & 0 & 1 & 0  
		\\ 0 & 0 & 0 & -1
		\\ 0 & -1 & 0 & 0
		\end{bmatrix}
		\begin{bmatrix}
		1 & 0 & 0 & 0
		\\ 0 & 1 & 0 & 0  
		\\ 0 & 0 & 1 & 0
		\\ 0 & 0 & 0 & 1
		\end{bmatrix}
		\nonumber \Rightarrow 
		\begin{bmatrix}
		1 & 0 & 0 & 0
		\\ 0 & -1 & 0 & 0
		\\ 0 & 0 & 1 & 0  
		\\ 0 & 0 & 0 & -1
		\end{bmatrix}
		\begin{bmatrix}
		1 & 0 & 0 & 0
		\\ 0 & 0 & 0 & 1
		\\ 0 & 1 & 0 & 0  
		\\ 0 & 0 & 1 & 0
		\end{bmatrix} \Rightarrow
		\\ 
		\begin{bmatrix}
		1 & 0 & 0 & 0
		\\ 0 & 1 & 0 & 0
		\\ 0 & 0 & 1 & 0  
		\\ 0 & 0 & 0 & 1
		\end{bmatrix}
		\begin{bmatrix}
		1 & 0 & 0 & 0
		\\ 0 & 0 & 0 & -1
		\\ 0 & 1 & 0 & 0  
		\\ 0 & 0 & -1 & 0
		\end{bmatrix}
		\end{multline}
		
		Así que  
		
		\begin{equation}
		[id]_{can}^S=\begin{pmatrix}
		1 & 0 & 0 & 0
		\\ 0 & 0 & 0 & -1
		\\ 0 & 1 & 0 & 0  
		\\ 0 & 0 & -1 & 0
		\end{pmatrix} \nonumber
		\end{equation}
		
		Entonces la representación matricial de $ g $ en esta base esta dada por
		\begin{eqnarray}
		[g]_S^S & = & [id]_{can}^S[g]_{can}^{can}[id]_S^{can}\nonumber
		\\ & = & 
		\begin{pmatrix}
		1 & 0 & 0 & 0
		\\ 0 & 0 & 0 & -1
		\\ 0 & 1 & 0 & 0  
		\\ 0 & 0 & -1 & 0
		\end{pmatrix}
		\begin{pmatrix}
		\frac{\sqrt{3}}{2} & 0 & 0 & \frac{1}{2}
		\\ 0 & \frac{\sqrt{3}}{2} & \frac{1}{2} & 0 
		\\ 0 & -\frac{1}{2} & \frac{\sqrt{3}}{2} & 0
		\\ -\frac{1}{2} & 0 & 0 & \frac{\sqrt{3}}{2}
		\end{pmatrix} 
		\begin{pmatrix}
		1 & 0 & 0 & 0
		\\ 0 & 0 & 1 & 0  
		\\ 0 & 0 & 0 & -1
		\\ 0 & -1 & 0 & 0
		\end{pmatrix} \nonumber
		\\ & = &
		\begin{pmatrix}
		1 & 0 & 0 & 0
		\\ 0 & 0 & 0 & -1
		\\ 0 & 1 & 0 & 0  
		\\ 0 & 0 & -1 & 0
		\end{pmatrix}
		\begin{pmatrix}
		\frac{\sqrt{3}}{2} & -\frac{1}{2} & 0 & 0
		\\ 0 & 0 & \frac{\sqrt{3}}{2} & -\frac{1}{2}  
		\\ 0 & 0 & -\frac{1}{2} & -\frac{\sqrt{3}}{2}
		\\ -\frac{1}{2} & -\frac{\sqrt{3}}{2} & 0 & 0
		\end{pmatrix} \nonumber
		\\ & = & 
		\begin{pmatrix}
		\frac{\sqrt{3}}{2} & -\frac{1}{2} & 0 & 0
		\\ \frac{1}{2} & \frac{\sqrt{3}}{2} & 0 & 0
		\\ 0 & 0 & \frac{\sqrt{3}}{2} & -\frac{1}{2}  
		\\ 0 & 0 & \frac{1}{2} & \frac{\sqrt{3}}{2}
		\end{pmatrix} \nonumber
		\end{eqnarray}
		
		Vemos que logramos diagonalizarla por bloques simples.
		
		Para $ h $ tenemos que la matriz de transformación de $ S $ a la inversa es
		
		\begin{equation}
		[id]_{S}^{can} = 
		\begin{pmatrix}
		0 & -1 & \frac{1}{2} & 0
		\\ 0 & 0 & \frac{1}{2} & 0
		\\ 0 & 0 & 0 & -\frac{1}{2}  
		\\ 1 & 0 & 0 & -\frac{1}{2}
		\end{pmatrix} \nonumber
		\end{equation}
		
		La matriz inversa de esta matriz la calculamos de nuevo usando Gauss-Jordan.
		
		\begin{multline}
		\begin{bmatrix}
		0 & -1 & \frac{1}{2} & 0
		\\ 0 & 0 & \frac{1}{2} & 0
		\\ 0 & 0 & 0 & -\frac{1}{2}  
		\\ 1 & 0 & 0 & -\frac{1}{2}
		\end{bmatrix}
		\begin{bmatrix}
		1 & 0 & 0 & 0
		\\ 0 & 1 & 0 & 0
		\\ 0 & 0 & 1 & 0  
		\\ 0 & 0 & 0 & 1
		\end{bmatrix} \nonumber \Rightarrow
		\begin{bmatrix}
		1 & 0 & 0 & -\frac{1}{2}
		\\ 0 & -1 & \frac{1}{2} & 0
		\\ 0 & 0 & \frac{1}{2} & 0
		\\ 0 & 0 & 0 & -\frac{1}{2}  
		\end{bmatrix}
		\begin{bmatrix}
		0 & 0 & 0 & 1
		\\ 1 & 0 & 0 & 0
		\\ 0 & 1 & 0 & 0
		\\ 0 & 0 & 1 & 0  		
		\end{bmatrix} \Rightarrow
		\\
		\begin{bmatrix}
		1 & 0 & 0 & 0
		\\ 0 & 1 & 0 & 0
		\\ 0 & 0 & 1 & 0
		\\ 0 & 0 & 0 & 1  
		\end{bmatrix}
		\begin{bmatrix}
 		0 & 0 & -1 & 1
 		\\ -1 & 1 & 0 & 0
 		\\ 0 & 2 & 0 & 0
 		\\ 0 & 0 & -2 & 0  		
 		\end{bmatrix}
		\end{multline}
		
		Entonces concluimos que
		
		\begin{equation}
		[id]_{can}^S = 
		\begin{pmatrix}
 		0 & 0 & -1 & 1
 		\\ -1 & 1 & 0 & 0
 		\\ 0 & 2 & 0 & 0
 		\\ 0 & 0 & -2 & 0  		
 		\end{pmatrix} \nonumber 
		\end{equation}
		
		Entonces la matriz de $ h $ en esta base esta dada por
		\begin{eqnarray}
		[h]_S^S & = & [id]_{can}^S[h]_{can}^{can}[id]_S^{can}\nonumber
		\\ & = & 
		\begin{pmatrix}
 		0 & 0 & -1 & 1
 		\\ -1 & 1 & 0 & 0
 		\\ 0 & 2 & 0 & 0
 		\\ 0 & 0 & -2 & 0  		
 		\end{pmatrix}
		\begin{pmatrix}
    	\frac{\sqrt{3}}{2} & 0 & \frac{1}{2} & 0
    	\\ 0 & \frac{\sqrt{3}}{2} & 1 & -\frac{1}{2} 
    	\\ \frac{1}{2} & -1 & \frac{\sqrt{3}}{2} & 0
    	\\ 0 & -\frac{1}{2} & 0 & \frac{\sqrt{3}}{2}
    	\end{pmatrix} 
		\begin{pmatrix}
		0 & -1 & \frac{1}{2} & 0
		\\ 0 & 0 & \frac{1}{2} & 0
		\\ 0 & 0 & 0 & -\frac{1}{2}  
		\\ 1 & 0 & 0 & -\frac{1}{2}
		\end{pmatrix} \nonumber
		\\ & = &
		\begin{pmatrix}
 		0 & 0 & -1 & 1
 		\\ -1 & 1 & 0 & 0
 		\\ 0 & 2 & 0 & 0
 		\\ 0 & 0 & -2 & 0  		
 		\end{pmatrix} 
		\begin{pmatrix}
		0 & -\frac{\sqrt{3}}{2} & \frac{\sqrt{3}}{4} & -\frac{1}{4}
		\\ -\frac{1}{2} & 0 & \frac{\sqrt{3}}{4} & -\frac{1}{4}
		\\ 0 & -\frac{1}{2} & -\frac{1}{4} & -\frac{\sqrt{3}}{4}  
		\\ \frac{\sqrt{3}}{2} & 0 & -\frac{1}{4} & -\frac{\sqrt{3}}{4}
		\end{pmatrix} \nonumber 
		\\ & = &
		\begin{pmatrix}
		\frac{\sqrt{3}}{2} & \frac{1}{2} & 0 & 0
		\\ -\frac{1}{2} & \frac{\sqrt{3}}{2} & 0 & 0
		\\ -1 & 0 & \frac{\sqrt{3}}{2} & -\frac{1}{2}  
		\\ 0 & 1 & \frac{1}{2} & \frac{\sqrt{3}}{2}
		\end{pmatrix} \nonumber 
		\end{eqnarray}
		
		Vemos que en este caso el resultado no fue una matriz diagonal por bloques.
		
		\item El polinomio minimal divide al polinomio caracteristico.
		Para $ g $ el polinomio minimal es $ (t-\frac{\sqrt{3}}{2})^2+\frac{1}{2}^2) = t^2 -\sqrt{3}t+1$.
		
		Para demostrar esto observese que 
	\end{enumerate}
	
	 
	\begin{exc}
		Suponga que $ f $ es semi-simple y sea $ V_1 \leq V $ un subespacio invariante bajo $ f $. Demuestre que la restricción de $ f $ a $ V_1, f_1 \in \Hom_K (V_1,V_1) $ es semi-simple.
	\end{exc}
	

	 
\end{exc}


\begin{proof}
		Suponga
\end{proof}	

\begin{exc}
	Suponga que $ f $ es semi-simple, demuestre que existe una descomposición:
	\[ V = V_1 \otimes \ldots \otimes V_r \]
	tal que, para $ i = 1, \ldots, r, V_i $ es invariante bajo $ f $ y la restricción de este a $ V_i, f_i \in \Hom_K (V_i, V_i) $ es simple. (\textit{Ayuda: } Use inducción fuerte en $ n = \dim_K (V) $), es decir asuma que el resultado es cierto para todo operador en un espacio de dimensión estrictamente menor que $ n $ y use el punto anterior).
\end{exc}

\begin{proof}
	Suponga que:
\end{proof}

\begin{exc}
	Demuestre que el polinomio minimal de un operador semi-simple es un producto de polinomios irreducibles ninguno de ellos repetido (es decir, elevados únicamente a la primera potencia; es decir, sin cuadrados que lo dividan) usando los siguientes pasos:
	\begin{enumerate}
		\item Usando la descomposición del punto anterior, demuestre que el polinomio minimal de $ f_i $ divide al polinomio minimal de $ f $
		\item Demuestre que el mínimo común múltiplos de los polinomios minimales de $ f_1, \ldots, f_r $  es el polinomio mínimal de $ f $. (\textit{Ayuda}: Sea $ P(t) $ este mínimo común múltiplo, dado $ v \in V $, tome $ v_i \in V_i, i = 1, \ldots, r, $ tales que $ v = v_1 + \ldots +v_r $, demuestre que $ P(f)(v) = P(f)(v_1) + \ldots + P(f)(v_r) = 0 + \ldots +0 = 0 $)
		\item Concluya. (\textit{Ayuda: El polinomio minimal de un operador simple es irreducible.})
	\end{enumerate}
\end{exc}

\textit{Solución}

\begin{exc}
	Identifique entre los operadores $ f,g $ de $ \RR^4 $ ya definidos, cual es semi-simple y cual no lo es. Justifique su respuesta.
	
\end{exc}

\textit{Solución}
\end{document}
